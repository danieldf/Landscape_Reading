%%%
% BeGiN
% Wed 19 Jun 2013 14:06:02 EDT
%%%
%% Presentation make-shift "class"...
% The paper and font size are chosen as with the 'Beamer' document
% class :: these sizes can be adjusted for various projector
% capabilities, including:
%   (*) 144mm:90mm (16:10),
%   (*) 120mm:96mm (16:9),
%   (*) 160mm:90mm (HDTV 720p/i),
%   (*) 192mm:108mm (HDTV 1080p/i), &
%   (*) 128mm:96mm (Beamer's default).
%   (*) 216mm:279mm (ANSI A papersize, aka 'letter')
\documentclass[       %
  paper=279mm:216mm,  % 'letter' size paper, landscape mode
  fontsize=12pt,      % 12pt font selection
  twoside,
  pagesize=auto,           % write page size to dvi or pdf
  version=last,
  numbers=noendperiod,% removes points for special parts (e.g. appendix)
  captions=nooneline, % do not distinguish between one or more lines in captions
  DIV=calc
]{scrartcl}
\usepackage{scrpage2}
\setkomafont{pagehead}{\normalfont\footnotesize\sffamily\bfseries}
\setkomafont{pagenumber}{\normalfont\footnotesize\sffamily\bfseries}
\deftripstyle{CPI}[0pt][0.5pt]{\rightmark}{}{\thepage}{}{}{}
\pagestyle{CPI}
\typearea[current]{calc}
\areaset[current]{\textwidth}{\textheight}
\usepackage{multicol}   % better column-balancing
\setlength{\columnsep}{1cm}
\setlength{\columnseprule}{0.4pt}
\setlength{\topmargin}{-2cm}
\setlength{\oddsidemargin}{0cm}
\setlength{\evensidemargin}{\oddsidemargin}
%
\usepackage{cmap}
\usepackage{datetime}
\renewcommand{\dateseparator}{-}
\settimeformat{hhmmsstime}
\newcommand{\todayiso}{\the\year \dateseparator \shortmonthname \dateseparator \twodigit\day}
\newcommand{\semver}{[DDF \href{http://semver.org/}{v1.0.0}]}
\usepackage{ae,aecompl,aeguill}
\usepackage{fix-cm}
\usepackage{lmodern}         % latin modern font
\usepackage[T1]{fontenc} % for correct hyphenation and T1 encoding
\usepackage[protrusion=true,expansion=false]{microtype} % for character protrusion and font expansion (only with pdflatex)
\usepackage[raise]{engord}
\usepackage{ucs}
\usepackage[utf8x]{inputenc}
\usepackage[x11names]{xcolor}
\usepackage{graphicx}
\usepackage{csquotes} % for inline quotations
\usepackage{ccicons}  % for CC licenses
\usepackage{tikz}     % sophisticated graphics package
\usepackage{mathtools}
\usepackage{dsfont,amstext,amssymb,amsbsy,amsopn,amsthm}
\usepackage[charter]{mathdesign}
\renewcommand{\sfdefault}{fvs}
\renewcommand{\ttdefault}{fvm}
\DeclareMathAlphabet{\mathpzc}{T1}{pzc}{m}{it}   % $\mathpzc{F}$
% \usepackage{bm}       % for bold math symbols
\usepackage{eucal}
\usepackage{empheq}
\usepackage{pifont}
\usepackage{textcomp}
\usepackage{wasysym}
\usepackage{calc}  % working with lengths, counters etc.
\usepackage[       %
  includeheadfoot, %
  vmargin=1cm,     %
  hmargin=2cm      %
]{geometry}        % set page layout parameters
\usepackage[3D]{movie15}
\usepackage{xkeyval}
\usepackage{lipsum}     % just some lorem-ipsum text filling
\usepackage{fancyvrb}
\usepackage[colorlinks=true,urlcolor=RoyalBlue3,linkcolor=OrangeRed3,citecolor=SpringGreen3,linktocpage=true]{hyperref}
\hypersetup{
  pdftitle    = {Digital Reading: two-column landscape format using KOMA scripts},
  pdfauthor   = {D.D. Ferrante <danieldf@het.brown.edu>},
  pdfsubject  = {eReading},
  pdfcreator  = {LaTeX2e with hyperref package},
  pdfproducer = {pdflatex},
  pdfkeywords = {Open Access, Open, Access, eReader, Digital, Reading, Online, LaTeX, KOMA},
  pdfview = {FitH},
  pdflang = {en_US}
}
\usepackage{bookmark}
\usepackage{cite}
%
%% extensible `=' sign ::  \stakrel{\text{definition}}{\hbox{\equalsfill}}
\makeatletter
\def\equalsfill{$\m@th\mathord=\mkern-7mu
  \cleaders\hbox{$\!\mathord=\!$}\hfill
  \mkern-7mu\mathord=$}
\makeatother
%% Hooked Square Root sign...
\def\hksqrt{\mathpalette\DHLhksqrt}
\def\DHLhksqrt#1#2{\setbox0=\hbox{$#1\sqrt{#2\,}$}\dimen0=\ht0
  \advance\dimen0-0.2\ht0
  \setbox2=\hbox{\vrule height\ht0 depth -\dimen0}%
{\box0\lower0.4pt\box2}}
%
\title{Online Reading: two-column landscape format using KOMA-scripts}
\author{\sffamily Daniel D. Ferrante\footnote{
  \href{http://www.het.brown.edu/people/danieldf/}{\copyright}\; \semver\,,\;
  \href{http://creativecommons.org/licenses/by-nc-sa/3.0/}{\ccbyncsa}\,.}}
\date{\sffamily\todayiso}
%
%%
\begin{document}
%
\maketitle
%
\begin{multicols}{2}
  \tableofcontents


  \section{Original Motivation}
  The original motivation came from
  ``\href{http://scholardox.wordpress.com/2013/05/29/two-column-landscape-should-be-the-standard-format-of-scholarly-online-articles/}
    {\emph{Advocating two-column landscape format for scholarly online articles}}''.
    
  However, because of typographical issues, among some other, I personally prefer
  to use the \href{http://www.ctan.org/pkg/koma-script}{KOMA-scripts}. Thus, I
  just wrote a \LaTeX\/ template that suited me better.


  \section{Text Example: \textit{Lorem Ipsum} and the Balancing of Columns}

  \lipsum
  

  \section{Multimedia Example: 3D Object and Control Toolbar}

  \includemovie[     %
    poster,          %
    toolbar,         % same as `controls'
    attach=true,     %
    label=brain.u3d, %
    text=(brain.u3d),%
    3Droo=42.473432080582924, %
	3Dcoo=-3.474954605102539 1.9501128196716309 6.208690643310547, %
	3Dc2c=-0.9117389917373657 -0.24471230804920197 -0.32992100715637207, %
	3Droll=-164.74804232815612, %
	3Dlights=CAD, %
	3Drender=SolidWireframe, %
  ]{\linewidth}{\linewidth}{brain.u3d}
  

  \section{Some Itemized Equations for Good Measure, and Some Mathematical Fonts}

  Before we proceed and show the code used to generate this file, let us create
  a list with some equations, just to see how things turn out:
  \begin{itemize}
  \item Let us open up the first item in a very cliché fashion with an inline
    equation: $E^2 = (m\, c^2)^2 + (p\, c)^2 \rightsquigarrow (\Box + m^2)\, \psi = 0$.
  \item Now we can move on to something more interesting, \emph{e.g.},
    \begin{align*}
      \mathscr{Z}(J) &= \int e^{i\, S(\phi,\, J)}\, \mathcal{D}\phi \;;\\
      \mathpzc{Z}{}_{\mathds{M}}[\mathfrak{J}]&= \varint_{\mathds{M}}
        e^{i\, S[\boldsymbol{\Phi};\, \mathfrak{J}]}\, \mathpzc{D}\boldsymbol{\Phi}\;.
    \end{align*}
  \end{itemize}

  
  \section{Ti\emph{k}Z/PGF Example}

\begin{center}
\begin{tikzpicture}[scale=1]
\foreach \t in {0,0.1,...,.8} {
\pgfmathsetmacro\x{6*\t^3 + 4 * 3 * \t^2*(1 - \t) + 2 * 3 * \t * (1 -\t)^2 + 0 * (1 - \t)^3}
\pgfmathsetmacro\z{1*\t^3 + 1 * 3 * \t^2*(1 - \t) + -1 * 3 * \t * (1 -\t)^2 + -1 * (1 - \t)^3 + 2}
\pgfmathsetmacro\tt{\t+.1}
\pgfmathsetmacro\xx{6*\tt^3 + 4 * 3 * \tt^2*(1 - \tt) + 2 * 3 * \tt * (1 -\tt)^2 + 0 * (1 - \tt)^3}
\pgfmathsetmacro\zz{1*\tt^3 + 1 * 3 * \tt^2*(1 - \tt) + -1 * 3 * \tt * (1 -\tt)^2 + -1 * (1 - \tt)^3 + 2}
% If second control is relative, it is relative to second end point!
\path[shade=axis, top color=green!50!black, bottom color=green!50!black, middle color=green]
 (\xx,0,-\zz)
 .. controls +(0,-.417,0) and +(0,.155,.387) ..
 ++(0,-0.928,-0.629)
 -- (\x, -0.928,-\z-.629)
 .. controls +(0,.155,.387) and +(0,-.417,0) ..
 ++(0,.928,.629)
 .. controls +(0,.555,0) and +(0,0,.555) ..
 ++(0,1,-1) 
 .. controls +(0,0,-.139) and +(0,.0516,.129) ..
 ++(0,-.072,-.371)
 -- (\xx,0.928,-\zz-2+.629)
 .. controls +(0,.0516,.129) and +(0,0,-.139) ..
 ++(0,.072,.371)
 .. controls +(0,0,.555) and +(0,.555,0) ..
 ++(0,-1,1);

\pgfmathsetmacro{\zz}{-\zz+2}
\pgfmathsetmacro{\z}{-\z+2}
\path[shade=axis, top color=green!50!black, bottom color=green!50!black, middle color=green]
 (\xx,0,-\zz)
 .. controls +(0,-.417,0) and +(0,.155,.387) ..
 ++(0,-0.928,-0.629)
 -- (\x, -0.928,-\z-.629)
 .. controls +(0,.155,.387) and +(0,-.417,0) ..
 ++(0,.928,.629)
 .. controls +(0,.555,0) and +(0,0,.555) ..
 ++(0,1,-1) 
 .. controls +(0,0,-.139) and +(0,.0516,.129) ..
 ++(0,-.072,-.371)
 -- (\xx,0.928,-\zz-2+.629)
 .. controls +(0,.0516,.129) and +(0,0,-.139) ..
 ++(0,.072,.371)
 .. controls +(0,0,.555) and +(0,.555,0) ..
 ++(0,-1,1);
}
\foreach \t in {0.8,0.9,...,1} {
\pgfmathsetmacro\x{6*\t^3 + 4 * 3 * \t^2*(1 - \t) + 2 * 3 * \t * (1 -\t)^2 + 0 * (1 - \t)^3}
\pgfmathsetmacro\z{1*\t^3 + 1 * 3 * \t^2*(1 - \t) + -1 * 3 * \t * (1 -\t)^2 + -1 * (1 - \t)^3 + 2}
\pgfmathsetmacro\tt{\t+.1}
\pgfmathsetmacro\xx{6*\tt^3 + 4 * 3 * \tt^2*(1 - \tt) + 2 * 3 * \tt * (1 -\tt)^2 + 0 * (1 - \tt)^3}
\pgfmathsetmacro\zz{1*\tt^3 + 1 * 3 * \tt^2*(1 - \tt) + -1 * 3 * \tt * (1 -\tt)^2 + -1 * (1 - \tt)^3 + 2}
%\path[shade=axis, top color=red, bottom color=black]
% If second control is relative, it is relative to second end point!
\path[shade=axis, top color=black, bottom color=black, middle color=red!70!black]
 (\xx,0,-\zz-2)
 .. controls +(0,-.555,0) and +(0,0,-.555) ..
 ++(0,-1,1)
 .. controls +(0,0,.139) and +(0,-.0516,-.129) ..
 ++(0,.072,.371)
 -- (\x,-0.928,-\z-.629)
 .. controls +(0,-.0516,-.129) and +(0,0,.139) ..
 ++(0,-.072,-.371)
 .. controls +(0,0,-.555) and +(0,-.555,0) ..
 ++(0,1,-1)
 .. controls +(0,+.417,0) and +(0,-.155,-.387) ..
 ++(0,+0.928,+0.629)
 -- (\xx, +0.928,-\zz-2+.629)
 .. controls +(0,-.155,-.387) and +(0,.417,0) ..
 ++(0,-.928,-.629)
 -- (\xx,0,-\zz-2);
\path[shade=axis, top color=green!50!black, bottom color=green!50!black, middle color=green]
 (\xx,0,-\zz)
 .. controls +(0,-.417,0) and +(0,.155,.387) ..
 ++(0,-0.928,-0.629)
 -- (\x, -0.928,-\z-.629)
 .. controls +(0,.155,.387) and +(0,-.417,0) ..
 ++(0,.928,.629)
 .. controls +(0,.555,0) and +(0,0,.555) ..
 ++(0,1,-1) 
 .. controls +(0,0,-.139) and +(0,.0516,.129) ..
 ++(0,-.072,-.371)
 -- (\xx,0.928,-\zz-2+.629)
 .. controls +(0,.0516,.129) and +(0,0,-.139) ..
 ++(0,.072,.371)
 .. controls +(0,0,.555) and +(0,.555,0) ..
 ++(0,-1,1);

\pgfmathsetmacro{\zz}{-\zz+2}
\pgfmathsetmacro{\z}{-\z+2}
% If second control is relative, it is relative to second end point!
\path[shade=axis, top color=black, bottom color=black, middle color=red!70!black]
 (\xx,0,-\zz-2)
 .. controls +(0,-.555,0) and +(0,0,-.555) ..
 ++(0,-1,1)
 .. controls +(0,0,.139) and +(0,-.0516,-.129) ..
 ++(0,.072,.371)
 -- (\x,-0.928,-\z-.629)
 .. controls +(0,-.0516,-.129) and +(0,0,.139) ..
 ++(0,-.072,-.371)
 .. controls +(0,0,-.555) and +(0,-.555,0) ..
 ++(0,1,-1)
 .. controls +(0,+.417,0) and +(0,-.155,-.387) ..
 ++(0,+0.928,+0.629)
 -- (\xx, +0.928,-\zz-2+.629)
 .. controls +(0,-.155,-.387) and +(0,.417,0) ..
 ++(0,-.928,-.629)
 -- (\xx,0,-\zz-2);
\path[shade=axis, top color=green!50!black, bottom color=green!50!black, middle color=green]
 (\xx,0,-\zz)
 .. controls +(0,-.417,0) and +(0,.155,.387) ..
 ++(0,-0.928,-0.629)
 -- (\x, -0.928,-\z-.629)
 .. controls +(0,.155,.387) and +(0,-.417,0) ..
 ++(0,.928,.629)
 .. controls +(0,.555,0) and +(0,0,.555) ..
 ++(0,1,-1) 
 .. controls +(0,0,-.139) and +(0,.0516,.129) ..
 ++(0,-.072,-.371)
 -- (\xx,0.928,-\zz-2+.629)
 .. controls +(0,.0516,.129) and +(0,0,-.139) ..
 ++(0,.072,.371)
 .. controls +(0,0,.555) and +(0,.555,0) ..
 ++(0,-1,1);
}
\end{tikzpicture}
\end{center}
  

  \section{Code Used to Generate this File}

  To compile this file you just need to run \texttt{pdflatex} on it as many
  times as necessary. Also, note that the order of some of the packages is
  important, for example, \texttt{hyperref} should be loaded before
  \texttt{bookmark}. Lastly, in order to access and make use of the multimedia
  file embedded, you need to use the \href{http://get.adobe.com/reader/}
  {Adobe Acrobat Reader}, for it seems to be the only one that implemented such
  features.
  
  \fvset{fontsize=\tiny,numbers=left,numbersep=2pt}
  \VerbatimInput{brain.tex}
\end{multicols}
%
%
\end{document}
%%%
% eNd
%%%

